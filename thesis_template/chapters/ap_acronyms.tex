\chapter{Acronyms}

This paragraph is an example of using the \verb|glossaries| package. The
\Gls{latex} typesetting markup language is especially suited for documents that
include \gls{maths}. \Glspl{formula} are rendered properly and easily, once one
gets used to the commands. Given a set of numbers, there are elementary methods
to compute its \acrlong{gcd}, which is abbreviated \acrshort{gcd}. This process
is similar to that used for the \acrfull{lcm}.

The acronyms are listed in Table~\ref{tab_acronymns} and the glossary
definitions are listed in Table~\ref{tab_glossary}.

% Print acronyms
\begin{table}[htbp!]
    \centering
    \caption{List of acronyms.}
    \printglossary[type=\acronymtype]
    \label{tab_acronymns}
\end{table}

% Print glossary
\begin{table}[htbp!]
    \centering
    \caption{List of glossary definitions.}
    \printglossary
    \label{tab_glossary}
\end{table}

If you would rather not use the \verb|glossaries| package, you can simply define macros using built-in \LaTeX{} commands:
\begin{verbatim}
    \def\gcd{Greatest Common Divisor}
\end{verbatim}