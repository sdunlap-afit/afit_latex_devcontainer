




%%%%%%%%%% Glossary Entries %%%%%%%%%%

% Example glossary and acronym definitions.
% TODO:
\newglossaryentry{latex}{name=\LaTeX{},
description={Is a markup language specially suited
for scientific documents}}
\newglossaryentry{maths}{name=mathematics,
description={Mathematics is what mathematicians do}}
\newglossaryentry{formula}{name=formula,
description={A mathematical expression}}


%%%%%%%%%% Optional Acronym Style %%%%%%%%%%
% 
% This is an optional acronym style that only shows the long (short) form for 
% the first use of an acronym, and only the long form for subsequent uses.
%
% Comment this out if you want the traditional long-short style (i.e., if you are a monster).
%
% For some reason, this needs to come after at least one glossary entry,
% but before any acronym definitions.
\newabbreviationstyle{long-short-noshort}{%
    \ifglsused{\glslabel}%
    {\GlsXtrUseAbbrStyleSetup{long-noshort}}%
    {\GlsXtrUseAbbrStyleSetup{long-short}}%    
}{%    
    \ifglsused{\glslabel}%
    {\GlsXtrUseAbbrStyleFmts{long-noshort}}%
    {\GlsXtrUseAbbrStyleFmts{long-short}}%    
}    

\setabbreviationstyle{long-short-noshort}


%%%%%%%%%% Acronym Entries %%%%%%%%%%

%\newabbreviation[description={}]{label}{short}{long}
% description - Shows up in table of acronyms
% label - internal label for referencing: use \gls{label}
% short - acronym
% long - full phrase (use upper/lower case as desired)

\newabbreviation[description={Least Common Multiple}]{lcm}{LCM}{least common multiple}
\newabbreviation[description={Greatest Common Divisor}]{gcd}{GCD}{greatest common divisor}
\newabbreviation[description={Global Navigation Satellite System}]{gnss}{GNSS}{global navigation satellite system}






